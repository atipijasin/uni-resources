\documentclass{article}
\usepackage[utf8]{inputenc}
\usepackage{hyperref}

\usepackage{graphicx}
\title{Appunti di Sistemi Operativi}
\author{Jasin Atipi}
\date{September 2017}

\begin{document}

\maketitle

\section*{1 Marzo 2017}
\subsection*{Processi e Thread (\href{https://didatticaonline.unitn.it/dol/pluginfile.php/368856/mod_resource/content/1/05-Processi_e_Thread.pdf}{slides})}
Il processo \`{e} l'istanza di un programma in esecuzione. Il programma \`{e} un concetto statico, il processo \`{e} dinamico. Il processo \`{e} eseguito in modo sequenziale, un'struzione alla volta. In un sistema multiprogrammato i processi evolvono in modo concorrente. Le risorse fisiche e logiche sono limitate. Il S.O. stesso \`{e} un insieme di processi.\\
Il processo consiste di:\\
$\bullet$ Istruzioni (sezione Codice o Testo). Parte statica del codice.\\
$\bullet$ Dati (sezione Dati). Variabili globali.\\
$\bullet$ Stack. Chiamate a procedura e parametri, variabili locali.\\
$\bullet$ Heap. Memoria allocata dinamica.\\
$\bullet$ Attributi (id, stato, controllo).\\
\\
\subsubsection*{Attributi (Process Control Block)}
All'interno del Sistema Operativo ogni processo \`{e} rappresentato dal PCB, che contiene informazioni importanti:\\
stato del processo, program counter, valori dei registri, finformazioni sulla memoria (registri limite, tabella pagine), informazioni sullo stato dell'I/O (richieste pendenti, file), informazioni sul'utilizzo del sistema (CPU), informazioni di scheduling.\\
\subsubsection*{Stato di un processo}
DUrante la sua exec., un processo evolve attraverso diversi stati. Diagrammi di stato diverso dipendentemente dal S.O.\\
Il processo pu\`{o} essere in exec o non in exec.\\
\subsubsection*{Scheduling}
Selezione del processo da eseguire nella CPU al fine di garantire: multiprogrammazione e time-sharing.\\
Long-term scheduler: seleziona quali processi devono essere trasferiti nella coda dei processi pronti. Ordine dei secondi\\
Short-term scheduler: seleziona quali sono i prossimi processi ad essere eseguiti ed alloca la CPU. Ordine dei millisecondi\\
Ogni processo \`{e} inserito in una serie di code: Coda di processi pronti o Coda di un dispositivo.\\
\subsubsection*{Dispatcher}
Cambio di contesto: salvataggio PCB del processo che esce e caricamento del PCB del processo che entra.\\
Passaggio alla modalit\`{a} utente: all'inizio della fase di dispatch il sistema si trova in modalit\`{a} kernel\\
Salto dell'istruzione da eseguire del processo appena arrivato nella CPU.\\
Il cambio di contesto (context switching) effettua il passaggio della CPU ad un nuovo processo.
\subsubsection*{Operazioni sui processi}
Un processo pu\`{o} creare un figlio. Il figlio ottiene le risorse dal S.O. o dal padre. I tipi di esecuzione sono di tipo sincrona (il padre attende la terminazione dei figli) o asincrona (evoluzione parallela o concorrente di padre e figli).\\
La creazione di un processo in UNIX avviene in 3 modi:\\
System call fork: crea un figlio che \`{e} un duplicato del padre.\\
System call exec: carica sul figlio un programma diverso da quello del padre.\\
System call wait: per esecuzione sincrona tra padre e figlio.\\



\end{document}
