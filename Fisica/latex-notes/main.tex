\documentclass{article}
\usepackage[utf8]{inputenc}
\usepackage{graphicx}
\everymath{\displaystyle}

\title{Appunti di Fisica}
\author{Jasin Atipi}
\date{September 2017}

\begin{document}

\maketitle

\section*{2 Marzo 2018}
\subsection*{Introduzione - Cinematica del punto}
Definizione di:\\
Accuratezza: taratura/calibrazione di uno strumento.\\
Precisione: risoluzione di uno strumento.
\subsubsection*{Moto degli oggetti}
Cominciamo a pensare alla posizione di un punto su una retta (orientata).
Abbiamo bisogno di un origine e una misura. La misura ci aiuta a definire la distanza di un punto
dall'origine. Per fare ci\`{o} usiamo un sistema di riferimento (retta) e un
sistema di misura (metri).\\
Una domanda importante \`{e}: cosa succede quando il punto si muove?
Dobbiamo introdurre il concetto di tempo.\\
Sia $x$ la posizione del punto (m lontano dall'origine) e $t$ il tempo in secondi.\\
\begin{tabular} {c | c}
    $t(s)$ & $x(m)$\\
    \hline
    0 & 2\\
    2 & 3,5\\
    5 & 3
\end{tabular}
Questo \`{e} un modo di definire una legge oraria, ovvero un mapping
$t\to x$, nel caso della tabella \`{e} di tipo discreto (non continuo).\\
Tutto ci\`{o} lo possiamo rappresentare in un piano cartesiano dove le ascisse
rappresentano il tempo $t(s)$ e le ordinate rappresentano la distanza dall'origine $x(m)$.\\
Nel caso della tabella si tratter\`{a} di un grafico discreto.\\
Un altro modo per defiire una legge oraria \`{e} in maniera analistica (funzione continua), endavremo una posizione definita tramite $x(t)$ ($x$ in funzione del tempo).
Per esempio $x(t)=22m$ rappresenta un punto fermo nel tempo (sempre 22 $m$).\\
Un altro esempio \`{e} $x(t)=5t$. Abbiamo un punto che si muove sempre di pi\`{u}
lontano dall'origine man mano che il tempo passa. In questa maniera so dove si trova 
continuamente il punto. Il numero 5 ha una dimensione. Dato che il $t$ \`{e}
espresso in secondi e $x(t)$ \`{e} espresso in metri, il prodotto delle dimensioni
di $5$ e $t$ deve restituire $m$. Quindi $5$ deve essere rappresentato in $\frac{m}{s}$.\\
Abbiamo quindi capito che il $5$ \`{e} una velocit\`{a}\\
Un punto fermo ha sempre velocit\`{a} 0.\\
Se prendiamo due posizioni $x_1, x_2$, possiamo determinare lo spostamento dell'oggetto
$\Delta x=x_2-x_1$. Il segno della velocit\`{a} indica se ci si sta spostando "avanti"
o "indietro" in base al passaggio del tempo.\\
Una importante osservazione da fare riguardo allo spostamento \`{e} 
che se si cambia l'origine della retta, lo spostamento di due punti $x_1, x_2$ rimane uguale.\\
Se usiamo un'origine $O$ in cui $x_1=3 m, x_2=5 m$, possiamo cambiare l'origine 
in $O'$ dove $x_1= 2m, x_4 = 4m$ e notiamo che $\Delta x= \Delta x' = x_2-x_1 = x_1'-x_2'$.\\
Questo ci aiuter\`{a} a definire la velocit\`{a} (spostamento nel tmepo).\\
\begin{tabular} {c | c}
    $t(s)$ & $x(m)$\\
    \hline
    0 & 1\\
    1 & 2\\
    4 & 1\\
    5 & 3,5
\end{tabular}
Possiamo dire che dall'istante 0 all'istante 1:\\
$\Delta x=1m, \Delta t=1s$\\
Possiamo anche dire che la velocit\`{a} media $\frac{\Delta x}{\Delta t}$\\
Nel nostro caso quindi la velocit\`{a} media $v_{media}=\frac{1m}{s}$.\\
La velocit\`{a} in cinematica ha dimensione $[V]=[LT^{-1}]$.\\
Possiamo calcolare la velocit\`{a} media per una coppia arbitraria di istanti, es:\\
$t=1$ e $t=4$\\
$v_{media}=\frac{(1-2)m}{(4-1)s}=-\frac{1m}{3s}\approx 0,33\frac{m}{s}$\\
Possiamo ricavare che $\Delta x = v_{media}\Delta t$.\\
\section*{7 Marzo 2018}
\subsection*{Moto rettilineo uniforme e moto rettilineo accellerato}
La velocit\`{a} pu\`{o} essere caratterizzata in velocit\`{a} media ed istantanea.\\
$v_{media}=\frac{\Delta x}{\Delta t}$.\\
$v$ (istantanea) $=\lim_{\Delta t\to 0}\frac{\Delta x}{\Delta t}=\frac{dx}{dt}$ (derivata in rispetto a $t$).\\
Consideriamo una $v$ costante, allora:\\
$v_{media}=\frac{\Delta x}{\Delta t} = v$, possiamo allora dire che $\Delta x = v\Delta t$.\\
Detto ci\`{o}, possiamo fare una legge oraria anche della velocit\`{a}, oltre che della posizione.\\
Data questa legge oraria, possiamo determinare lo spostamento $\Delta x$ tramite il prodotto $v\cdot \Delta t$, che corrisponde
all'area sottesa al grafico nell'intervallo di tempo voluto.\\
Sia $x-x_0$ lo spostamento, allora $x-x_0=v(t-t_0)$. A questo punto possiamo riscrivere
l'equazione:\\
$x(t)=x_0+v(t-t_0)$\\
A partire dalla posizione iniziale, un oggetto si sposta unofromemente di $v(t-t_0)$,
ma tale equazione rappresenta una retta ($x_0$ \`{e} l'intercetta)
e $v$ \`{e} il coefficiente angolare.\\
Se $t=t_0\Rightarrow x(t_0)=x_0$. Il segno di $v$ indica la direzione del moto.\\
Es con velocit\`{a} costante:\\
Due gareggiatori corrono dritti su una pedana a velocit\`{a} diverse\\
Partenza: $t_0= 0 s, x_0= 0 m$. In questo caso partono nello stesso istante e 
dalla stessa posizione, quindi vincer\`{a} quello pi\`{u} veloce.\\
Assumiamo ora che partono da posizioni diverse. Quello pi\`{u} veloce parte da 
una posizione minore di quello pi\`{u} lento. Date queste assunzioni, vi sar\`{a}
un momento in cui i due gareggiatori si incontrano.\\
$x_1(t)=0+v_1 t$ (quello pi\`{u} veloce parte dall'origine)\\
$x_2(t)=x_v+v_2t$ ($x_v$ vantaggio)\\
Il sorpasso avviene quando \\$x_1(t_s)=x_2(t_s)$ (sorpasso)\\
$v_1t_s=x_v+v_2t_s$\\
$t_s(v_1-v_2)=x_v$\\
$t_s=\frac{x_v}{v_1-v_2}$\\
Controlliamo le dimensioni:\\
$[T]=\frac{[L]}{[LT^{-1}]}=[T]$\\
Una volta fatti dei ragionamenti per verificare la correttezza
della formula, possiamo iniziare a mettere dentro i numeri.\\
Per esempio supponiamo che:\\
$v_1=5,2m\slash s$\\
$v_2=2,6m\slash s$\\
$x_v=12,0 m$\\
$t_s=\frac{12,0m}{2,4m\slash s}=5,0 s$\\
Il tempo di sorpasso in questo caso \`{e} 5 secondi.\\
(Rifare l'esercizio in cui c\`{e} un vantaggio nel tempo.)\\
\\
Con velocit\`{a} costante abbiamo le seguenti equazioni:\\
Posizione: $x(t)=x_0+v(t-t_0)$\\
Velocit\`{a}: $v(t)=v$ (costante)\\
La variazione di velocit\`{a} si chiama accelerazione. In un
istante di tempo $t_1$ abbiamo velocit\`{a} $v_1$ e in un altro istante
$t_2$ abbiamo $v_2$.\\
$\frac{\Delta v}{\Delta t}=\frac{v_2-v_1}{t_2-t_1}=v$ media\\
L'accelerazione ha dimensioni:\\
$\frac{[V]}{[T]}=[LT^{-2}]$\\
$a=\lim_{\Delta t\to 0}\frac{\Delta v}{\Delta t}=\frac{dv}{dx}$ (derivata della
velocit\`{a}).\\
Dato il grafico di una velocit\`{a} costante, lo spazio percorso
\`{e} rappresentato dall'area sottesa al grafico (un rettangolo). Si tratta
quindi dell'integrale:\\
$\int_{t_0}^{t}v(t) dt$\\
Dunque abbiamo che la posizione in base alla velocit\`{a} (nel moto rettilineo uniforme) \`{e} ricavabile da:\\
$x(t)=x_0+\int_{t_0}^{t}v(t) dt$\\
E la velocit\`{a} in base all'accelerazione \`{e} ricabaile da:\\
$v(t)=v_0+\int_{t_0}^{t}a(t) dt$\\
Ma nel moto uniforme accellerato abbiamo che l'accelerazione
\`{e}' costante, quindi:\\
$v(t)=v_0+a(t-t_0)$\\
\section*{9 Marzo 2017}
Dato un punto che si muove su una retta orientata abbiamo stabilito la 
posizione, velocit\`{a} e accelerazione (e come passare da una al'altra).\\
Abbiamo visto il moto rettilineo uniforme, dove $v=$costante, e $x = x_0 + v(t-t_0)$\\
Poi abbiamo visto il moto uniforme accellerato con velocit\`{a} non costante, ma variabile linearmente, dove $v=v_0+a(t-t_0)$.\\
Si pu\`{o} ricavare la posizione dall'accelerazione:\\
$x=x_0+\int_{t_0}^t v(t)dt=x_0+v_0t+\frac{1}{2}at^2$\\
\subsection*{Caduta libera}
Nella caduta libera siamo in presenza di un MUA, dove l'accelerazione 
(terrestre) \`{e} $g=9,8m\slash s$\\
Sia $y_0=0$ l'origine (quota di partenza) e $a=g$\\
Sia $H=18m$ la profondit\`{a} di un pozzo. Lasciando cadere un sasso
in questo pozzo, quanto ci mette a toccare l'acqua in fondo?\\
Sia $t_c$ il tempo di caduta, allora $H=\frac{1}{2}gt_c^2\Rightarrow 
t_c=\sqrt{\frac{2H}{g}}$.\\
Nel nostro caso $H=18,0m$, quindi:\\
$t_c=\sqrt{\frac{2\cdot 18,0}{9,8}}\approx1,9 s$\\



\end{document}
