\documentclass{article}
\usepackage[utf8]{inputenc}
\usepackage{graphicx}
\usepackage{mathtools}

\everymath{\displaystyle}

\title{Appunti di Fisica}
\author{Jasin Atipi}
\date{September 2017}

\begin{document}

\maketitle

\section*{2 Marzo 2018}
\subsection*{Introduzione - Cinematica del punto}
Definizione di:\\
Accuratezza: taratura/calibrazione di uno strumento.\\
Precisione: risoluzione di uno strumento.
\subsubsection*{Moto degli oggetti}
Cominciamo a pensare alla posizione di un punto su una retta (orientata).
Abbiamo bisogno di un origine e una misura. La misura ci aiuta a definire la distanza di un punto
dall'origine. Per fare ci\`{o} usiamo un sistema di riferimento (retta) e un
sistema di misura (metri).\\
Una domanda importante \`{e}: cosa succede quando il punto si muove?
Dobbiamo introdurre il concetto di tempo.\\
Sia $x$ la posizione del punto (m lontano dall'origine) e $t$ il tempo in secondi.\\
\begin{tabular} {c | c}
    $t(s)$ & $x(m)$\\
    \hline
    0 & 2\\
    2 & 3,5\\
    5 & 3
\end{tabular}
Questo \`{e} un modo di definire una legge oraria, ovvero un mapping
$t\to x$, nel caso della tabella \`{e} di tipo discreto (non continuo).\\
Tutto ci\`{o} lo possiamo rappresentare in un piano cartesiano dove le ascisse
rappresentano il tempo $t(s)$ e le ordinate rappresentano la distanza dall'origine $x(m)$.\\
Nel caso della tabella si tratter\`{a} di un grafico discreto.\\
Un altro modo per defiire una legge oraria \`{e} in maniera analistica (funzione continua), endavremo una posizione definita tramite $x(t)$ ($x$ in funzione del tempo).
Per esempio $x(t)=22m$ rappresenta un punto fermo nel tempo (sempre 22 $m$).\\
Un altro esempio \`{e} $x(t)=5t$. Abbiamo un punto che si muove sempre di pi\`{u}
lontano dall'origine man mano che il tempo passa. In questa maniera so dove si trova 
continuamente il punto. Il numero 5 ha una dimensione. Dato che il $t$ \`{e}
espresso in secondi e $x(t)$ \`{e} espresso in metri, il prodotto delle dimensioni
di $5$ e $t$ deve restituire $m$. Quindi $5$ deve essere rappresentato in $\frac{m}{s}$.\\
Abbiamo quindi capito che il $5$ \`{e} una velocit\`{a}\\
Un punto fermo ha sempre velocit\`{a} 0.\\
Se prendiamo due posizioni $x_1, x_2$, possiamo determinare lo spostamento dell'oggetto
$\Delta x=x_2-x_1$. Il segno della velocit\`{a} indica se ci si sta spostando "avanti"
o "indietro" in base al passaggio del tempo.\\
Una importante osservazione da fare riguardo allo spostamento \`{e} 
che se si cambia l'origine della retta, lo spostamento di due punti $x_1, x_2$ rimane uguale.\\
Se usiamo un'origine $O$ in cui $x_1=3 m, x_2=5 m$, possiamo cambiare l'origine 
in $O'$ dove $x_1= 2m, x_4 = 4m$ e notiamo che $\Delta x= \Delta x' = x_2-x_1 = x_1'-x_2'$.\\
Questo ci aiuter\`{a} a definire la velocit\`{a} (spostamento nel tmepo).\\
\begin{tabular} {c | c}
    $t(s)$ & $x(m)$\\
    \hline
    0 & 1\\
    1 & 2\\
    4 & 1\\
    5 & 3,5
\end{tabular}
Possiamo dire che dall'istante 0 all'istante 1:\\
$\Delta x=1m, \Delta t=1s$\\
Possiamo anche dire che la velocit\`{a} media $\frac{\Delta x}{\Delta t}$\\
Nel nostro caso quindi la velocit\`{a} media $v_{media}=\frac{1m}{s}$.\\
La velocit\`{a} in cinematica ha dimensione $[V]=[LT^{-1}]$.\\
Possiamo calcolare la velocit\`{a} media per una coppia arbitraria di istanti, es:\\
$t=1$ e $t=4$\\
$v_{media}=\frac{(1-2)m}{(4-1)s}=-\frac{1m}{3s}\approx 0,33\frac{m}{s}$\\
Possiamo ricavare che $\Delta x = v_{media}\Delta t$.\\
\section*{7 Marzo 2018}
\subsection*{Moto rettilineo uniforme e moto rettilineo accellerato}
La velocit\`{a} pu\`{o} essere caratterizzata in velocit\`{a} media ed istantanea.\\
$v_{media}=\frac{\Delta x}{\Delta t}$.\\
$v$ (istantanea) $=\lim_{\Delta t\to 0}\frac{\Delta x}{\Delta t}=\frac{dx}{dt}$ (derivata in rispetto a $t$).\\
Consideriamo una $v$ costante, allora:\\
$v_{media}=\frac{\Delta x}{\Delta t} = v$, possiamo allora dire che $\Delta x = v\Delta t$.\\
Detto ci\`{o}, possiamo fare una legge oraria anche della velocit\`{a}, oltre che della posizione.\\
Data questa legge oraria, possiamo determinare lo spostamento $\Delta x$ tramite il prodotto $v\cdot \Delta t$, che corrisponde
all'area sottesa al grafico nell'intervallo di tempo voluto.\\
Sia $x-x_0$ lo spostamento, allora $x-x_0=v(t-t_0)$. A questo punto possiamo riscrivere
l'equazione:\\
$x(t)=x_0+v(t-t_0)$\\
A partire dalla posizione iniziale, un oggetto si sposta unofromemente di $v(t-t_0)$,
ma tale equazione rappresenta una retta ($x_0$ \`{e} l'intercetta)
e $v$ \`{e} il coefficiente angolare.\\
Se $t=t_0\Rightarrow x(t_0)=x_0$. Il segno di $v$ indica la direzione del moto.\\
Es con velocit\`{a} costante:\\
Due gareggiatori corrono dritti su una pedana a velocit\`{a} diverse\\
Partenza: $t_0= 0 s, x_0= 0 m$. In questo caso partono nello stesso istante e 
dalla stessa posizione, quindi vincer\`{a} quello pi\`{u} veloce.\\
Assumiamo ora che partono da posizioni diverse. Quello pi\`{u} veloce parte da 
una posizione minore di quello pi\`{u} lento. Date queste assunzioni, vi sar\`{a}
un momento in cui i due gareggiatori si incontrano.\\
$x_1(t)=0+v_1 t$ (quello pi\`{u} veloce parte dall'origine)\\
$x_2(t)=x_v+v_2t$ ($x_v$ vantaggio)\\
Il sorpasso avviene quando \\$x_1(t_s)=x_2(t_s)$ (sorpasso)\\
$v_1t_s=x_v+v_2t_s$\\
$t_s(v_1-v_2)=x_v$\\
$t_s=\frac{x_v}{v_1-v_2}$\\
Controlliamo le dimensioni:\\
$[T]=\frac{[L]}{[LT^{-1}]}=[T]$\\
Una volta fatti dei ragionamenti per verificare la correttezza
della formula, possiamo iniziare a mettere dentro i numeri.\\
Per esempio supponiamo che:\\
$v_1=5,2m\slash s$\\
$v_2=2,6m\slash s$\\
$x_v=12,0 m$\\
$t_s=\frac{12,0m}{2,4m\slash s}=5,0 s$\\
Il tempo di sorpasso in questo caso \`{e} 5 secondi.\\
(Rifare l'esercizio in cui c\`{e} un vantaggio nel tempo.)\\
\\
Con velocit\`{a} costante abbiamo le seguenti equazioni:\\
Posizione: $x(t)=x_0+v(t-t_0)$\\
Velocit\`{a}: $v(t)=v$ (costante)\\
La variazione di velocit\`{a} si chiama accelerazione. In un
istante di tempo $t_1$ abbiamo velocit\`{a} $v_1$ e in un altro istante
$t_2$ abbiamo $v_2$.\\
$\frac{\Delta v}{\Delta t}=\frac{v_2-v_1}{t_2-t_1}=v$ media\\
L'accelerazione ha dimensioni:\\
$\frac{[V]}{[T]}=[LT^{-2}]$\\
$a=\lim_{\Delta t\to 0}\frac{\Delta v}{\Delta t}=\frac{dv}{dx}$ (derivata della
velocit\`{a}).\\
Dato il grafico di una velocit\`{a} costante, lo spazio percorso
\`{e} rappresentato dall'area sottesa al grafico (un rettangolo). Si tratta
quindi dell'integrale:\\
$\int_{t_0}^{t}v(t) dt$\\
Dunque abbiamo che la posizione in base alla velocit\`{a} (nel moto rettilineo uniforme) \`{e} ricavabile da:\\
$x(t)=x_0+\int_{t_0}^{t}v(t) dt$\\
E la velocit\`{a} in base all'accelerazione \`{e} ricabaile da:\\
$v(t)=v_0+\int_{t_0}^{t}a(t) dt$\\
Ma nel moto uniforme accellerato abbiamo che l'accelerazione
\`{e}' costante, quindi:\\
$v(t)=v_0+a(t-t_0)$\\
\section*{9 Marzo 2017}
Dato un punto che si muove su una retta orientata abbiamo stabilito la 
posizione, velocit\`{a} e accelerazione (e come passare da una al'altra).\\
Abbiamo visto il moto rettilineo uniforme, dove $v=$costante, e $x = x_0 + v(t-t_0)$\\
Poi abbiamo visto il moto uniforme accellerato con velocit\`{a} non costante, ma variabile linearmente, dove $v=v_0+a(t-t_0)$.\\
Si pu\`{o} ricavare la posizione dall'accelerazione:\\
$x=x_0+\int_{t_0}^t v(t)dt=x_0+v_0t+\frac{1}{2}at^2$\\
\subsection*{Caduta libera}
\subsubsection*{Esempio 1}
Nella caduta libera siamo in presenza di un MUA, dove l'accelerazione 
(terrestre) \`{e} $g=9,8m\slash s$\\
Sia $y_0=0$ l'origine (quota di partenza) e $a=g$\\
Sia $H=18m$ la profondit\`{a} di un pozzo. Lasciando cadere un sasso
in questo pozzo, quanto ci mette a toccare l'acqua in fondo?\\
Sia $t_c$ il tempo di caduta, allora $H=\frac{1}{2}gt_c^2\Rightarrow 
t_c=\sqrt{\frac{2H}{g}}$.\\
Nel nostro caso $H=18,0m$, quindi:\\
$t_c=\sqrt{\frac{2\cdot 18,0}{9,8}}\approx1,9 s$\\
\subsubsection*{Esenpio 2}
Cosa succede quando lancio un oggetto verso l'alto (sulla terra)?\\
Abbiamo un'accelerazione che \`{e} pari a $-g$ (forza di gravit\`{a}) che agisce
verso il basso dato che tiro l'oggetto verso l'alto.\\
Quindi:\\
$
\begin{cases}
a=-gt\\
v=v_0-gt\\
y=0+v_0t-\frac{1}{2}gt^2
\end{cases}
$
\\
L'oggetto \`{e} al "suolo" quando $t=0$ e $t=\frac{2v_0}{g}$ perch\'{e}
$y=v_0t-\frac{1}{2}gt^2\Leftrightarrow y=t(v_0-\frac{1}{2}gt^2$ (risolvendola per $t$ e mettendola
$=0$ si trovano le soluzionio in cui vale 0)\\
La veclocit\`{a} \`{e} a 0 a met\`{a} dei momenti in cui \`{e} al suolo, quindi:\\
$(0+\frac{2v_0}{g})\frac{1}{2}$\\
$v_0=0\Leftrightarrow \exists t_m=\frac{v_0}{g}$\\
Quando torna al suolo la velocit\`{a} \`{e} uguale a $-v_0$\\
$t_f=\frac{2v_0}{g}$\\
$v=v_0-g\cdot\frac{2v_0}{g}=-v_0$\\
L'altezza massima \`{e} quando $t=t_M=\frac{v_0}{g}$\\
$y=v_0\cdot\frac{v_0}{g}-\frac{1}{2}gv\frac{v_0^2}{g^2}\Leftrightarrow$\\
$y=\frac{v_0^2}{g}-\frac{1}{2}\frac{v_0^2}{g}=\frac{v_0^2}{2g}$\\
La dimensione di questa formula \`{e} $[\frac{L^2T^2}{LT^{-2}}=[L]$ che \`{e}
una lunghezza ed \`{e} ci\`{o} che cerchiamo.\\
\subsubsection*{Esercizio da fare}
Si lanciano due oggetti verso l'alto in tempi diversi ma tutto il resto uguale,
calcolare la formula per determinare il momento in cui si incontrano.\\
\section*{12 Marzo 2018}
\subsection*{Cinematica del punto in 2d}
In 2 dimensioni, un punto si move in un piano cartesiano (assi $x$ e $y$). La posizione
del punto nel tempo \`{e} definita da un vettore $\vec{r}=(x(t),y(t))$. $\vec{r}$ 
\`{e} un vettore a 2 componenti (in questo caso).\\
La lunghezza (o modulo o intensit\`{a}) di $\vec{r}$ si indica con $r$ e corrisponde
alla lunghezza della freccia che rappresenta $\vec{r}$, o alla distanza del punto rappresentato
da $\vec{r}$ dall'origine.\\
$r=\sqrt{x^2+y^2}$.\\
Quando il punto si sposta, \`{e} come se spostassi la freccia (con coda nell'origine).
Lo spostamento \`{e} rappresentato anch'esso da una freccia, che parte dalla testa
della prima freccia (non spostata) e si ferma alla testa della seconda freccia (spostata).
Tale vettore di spostamento \`{e} indicato e rappresentato da: $\Delta \vec{r}=(\Delta x,\Delta y)$\\
Ricordiamo che $v_M=\frac{\Delta x}{\Delta t}$. In $2D$, la velocit\`{a} media \`{e} anch'essa un vettore. Ha quindi
una direzione e una lunghezza, ed \`{e} rappresentata da due componenti: $\vec{v}_M=\frac{\Delta \vec{r}}{\Delta t}=(\frac{\Delta x}{\Delta t},\frac{\Delta y}{\Delta t})$\\
Tale $v_M$ non ci dice cosa succede in ogni istante del movimento.\\
Per ottenere la velocit\`{a} in ogni istante del movimento, bisogna eseguire il rapporto
incrementale su $\vec{r}$.\\
Ricordiamo che lo spostamento del punto \`{e} rappresentato dallo spostamento del vettore
da un punto ad un altro che produce un terzo vettore di spostamento $\Delta \vec{r}$.\\
Facciamo tendere tale $\Delta \vec{r}$ a 0. $\Delta r$ \`{e} il modulo di $\Delta \vec{r}$.\\
Per $\Delta \vec{r}\to 0$, $\Delta r\to ds$ dove s sta per "strada". La strada $\Delta s$ rappresenta 
la strada che si percorre dato uno spostamento (mentre $\Delta r$ rappresenta 
la distanza da un punto all'altro, dato uno spostamento, che in generale non \`{e}
uguale alla strada percorsa, ma per $\Delta r\to 0$, diventa uguale).\\
Si pu\`{o} inoltre dimostrare che per $\Delta t\to 0, \Delta \vec{r}$ tende alla
traiettoria del percorso in un arbitrario momento di tempo.\\
Abbiamo dunque che per $\Delta t\to 0, \vec{v}_M\to\vec{v}$.\\
$\vec{v}$ \`{e} la velocit\`{a} vettoriale (velocity), e bisogna differirla dalla 
velocit\`{a} scalare (speed) $v=\frac{ds}{dt}$\\
Un fatto importante \`{e} che $\vec{v}$ \`{e} un vettore tangente alla traiettoria
in un momento arbitrario del movimento del punto.\\
Si pu\`{o} fare un passo avanti e calcolare anche l'accelerazione, possiamo riprendere
le nozioni dell'accelerazione in 1D e usarle per rappresentare l'accelerazione in 2D.\\
Abbiamo che $\vec{a}_M=(\frac{\Delta v_x}{\Delta t}, \frac{\Delta v_y}{\Delta t})=\frac{\Delta \vec{v}}{\Delta t}$\\
Nasce per\`{o} una nuova idea di accelerazione:\\
$\vec{a}=\lim_{\Delta t\to 0}\vec{a}_M=\lim_{\Delta t\to 0}\frac{\Delta \vec{v}}{\Delta t}$\\
Studiamo i seguenti 4 casi:\\
$a$) Moto rettilineo uniforme (punto che si muove dritto con la stessa velocit\`{a}): abbiamo che l'accelerazione \`{e} doppiamente nulla\\
$b$) Moto rettilineo non uniforme (punto che si muove dritto con velocit\`{a} variabile): in questo caso l'accelerazione $a\neq 0$ (con $a=|\vec{a}|$) ma
$\vec{a}=0$.\\
$c$) Moto curvilineo uniforme (punto che si muove curvando la stessa velocit\`{a}): in questo caso abbiamo che $a=0$ e $\vec{a}\neq 0$\\
$d$) Modo curvilineo non uniforme (punto che si muove curvando con velocit`{a} variabile): in questo caso entrambe le accelerazioni sono diverse da zero: $a\neq 0, \vec{a}\neq 0$.\\
Se prendo una traiettoria qualsiasi, e un punto qualsiasi in tale
traiettoria, $\exists$ un cerchio con raggio $R$ che \`{e} tangente in quel punto. 
La curvatura in un punto si indica con il reciproco del raggio $\frac{1}{R}$ di tale cerchio.\\
Ovviamente un cerchio con raggio molto grande indica una curvatura
assai piccola e viceversa.\\
Se ci si muove su una curva a forma di cerchio, la curvatura in ogni punto
\`{e} rappresentata dal cerchio stesso. Se vado a velocit\`{a} $v$ costante, abbiamo che
$a=0,\vec{a}\neq 0$. In tal caso $\vec{a}$ si chiama accelerazione centripeta
poich\`{e} in ogni momento il punto cerca di andare verso il centro
del cerchio. Si pu\`{o} dimostrare l'accelerazione di sterzo si pu\`{o}
calcolare tramite $\frac{v^2}{R}$.\\
Abbiamo che:\\
$\vec{a}=\vec{a}_{tangenziale}+\vec{a}_{centripeta}$ in quasiasi monento.\\
$|\vec{a}_{tangenziale}|=\frac{dv}{dt}$\\
$|\vec{a}_{centripeta}|=\frac{v^2}{R}$\\
Dunque in un qualsiasi momento del moto del punto, abbiamo:\\
$\vec{v}, \vec{a}_{tangenziale}, \vec{a}_{centripeta}$ e $\vec{a} = \vec{a}_{tangenziale}+ \vec{a}_{centripeta}$\\
\subsubsection*{Lancio di un oggetto in traiettoria parabolica}
Un oggetto lanciato in traiettoria parabolica ha un moto orizzontale e verticale
che si possono separare.\\
Abbiamo che:\\
$v_{0x}=v_0\cos\theta$\\
$v_{0y}=v_0\sin\theta$\\
Dove $\theta$ \`{e} l'angolo con cui lanciamo l'oggetto.\\
Studiamo il moto dell'oggetto separando il moto orizzontale da quello verticale:\\
$x:\\
a_x=0\\
v_x=v_{0x}+a_xt=v_{0x}\\
x=0+v_{0x}t+\frac{1}{2}a_xt^2=v_{0x}t\\
\\
y:\\
a_y=-g\\
v_y=v_{0y}+a_yt=v_{0y-gt}\\
y=0+v_{0y}t+\frac{1}{2}a_yt^2=v_{0x}t$
Sia $t_M$ il momento in cui l'oggetto ha altezza orizzontale massima, sappiamo che:\\
$t_M=\frac{v_{0y}}{g}$\\
Allora il tempo totale di volo $t_V=2\frac{v_{0y}}{g}$\\
Cerchiamo di capire la gittata (quanto va lontano).\\
$G=v_{0x}t_v$\\
Notiamo che la dimensione \`{e} giusta:\\
$[\frac{L^2T^{-2}}{LT^2}]=[L]$\\
Possiamo riscrivere la formula della gittata in questo modo:\\
$\frac{v_0^2\sin\theta\cos\theta}{g}=\frac{v_0^2\sin2\theta}{g}$\\
Tale formula prende valore massimo (dati velocit\`{a} e accelerazione) quando $\theta=\frac{\pi}{4}=45^\circ$\\
Possiamo dimostrare che la traiettoria ha forma parabolica:\\
$\begin{cases}
x=v_{0x}t\\
y=v_{0y}t-\frac{1}{2}gt^2
\end{cases}$\\
$t=\frac{x_0}{v_{0x}}\Rightarrow y=\frac{v_0y}{v_0x}x-\frac{1}{2}g\frac{x^2}{v_{0x}^2}$\\
Tale formula \`{e} di forma $Ax+Bx^2$ che rappresenta una parabola.\\
\subsubsection*{Moto circoferenza di raggio R a velocit\`{a} costante}
Abbaimo che $\vec{v}$ \`{e} proporzionale al raggio $R$.\\
La posizione \`{e} rappresentata da $\vec{r}=(x,y)$\\
Ma per rappresentare la posizione si pu\`{o} anche usare l'angolo $\theta$ e la distanza
dall'origine ($R$).\\
Abbiamo quindi che dato un angolo $\theta$:\\
$x=R\cos\theta$\\
$y=R\sin\theta$\\
$v_x=\frac{dx}{dt}=-R\sin\theta\cdot\frac{d\theta}{dt}=-R\omega\sin\theta$\\
$v_y=\frac{dy}{dt}=R\cos\theta\cdot\frac{d\theta}{dt}=R\omega\cos\theta$\\
La strada $s$ si ottiene $s=R\theta\Rightarrow\frac{ds}{dt}=v=R\frac{d\theta}{dt}=R\omega$\\
$s=s_0+vt\Leftrightarrow R\theta=R\theta_0+vt\Leftrightarrow
\theta=\theta_0+\frac{vt}{R}\Leftrightarrow\theta=\theta_0+\omega t \\(\omega=\frac{v}{R})$
\end{document}
