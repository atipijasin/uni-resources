\documentclass{article}
\usepackage[utf8]{inputenc}
\usepackage{graphicx}
\everymath{\displaystyle}

\title{Appunti di Fisica}
\author{Jasin Atipi}
\date{September 2017}

\begin{document}

\maketitle

\section*{2 Marzo 2018}
\subsection*{Introduzione - Cinematica del punto}
Definizione di:\\
Accuratezza: taratura/calibrazione di uno strumento.\\
Precisione: risoluzione di uno strumento.
\subsubsection*{Moto degli oggetti}
Cominciamo a pensare alla posizione di un punto su una retta (orientata).
Abbiamo bisogno di un origine e una misura. La misura ci aiuta a definire la distanza di un punto
dall'origine. Per fare ci\`{o} usiamo un sistema di riferimento (retta) e un
sistema di misura (metri).\\
Una domanda importante \`{e}: cosa succede quando il punto si muove?
Dobbiamo introdurre il concetto di tempo.\\
Sia $x$ la posizione del punto (m lontano dall'origine) e $t$ il tempo in secondi.\\
\begin{tabular} {c | c}
    $t(s)$ & $x(m)$\\
    \hline
    0 & 2\\
    2 & 3,5\\
    5 & 3
\end{tabular}
Questo \`{e} un modo di definire una legge oraria, ovvero un mapping
$t\to x$, nel caso della tabella \`{e} di tipo discreto (non continuo).\\
Tutto ci\`{o} lo possiamo rappresentare in un piano cartesiano dove le ascisse
rappresentano il tempo $t(s)$ e le ordinate rappresentano la distanza dall'origine $x(m)$.\\
Nel caso della tabella si tratter\`{a} di un grafico discreto.\\
Un altro modo per defiire una legge oraria \`{e} in maniera analistica (funzione continua), endavremo una posizione definita tramite $x(t)$ ($x$ in funzione del tempo).
Per esempio $x(t)=22m$ rappresenta un punto fermo nel tempo (sempre 22 $m$).\\
Un altro esempio \`{e} $x(t)=5t$. Abbiamo un punto che si muove sempre di pi\`{u}
lontano dall'origine man mano che il tempo passa. In questa maniera so dove si trova 
continuamente il punto. Il numero 5 ha una dimensione. Dato che il $t$ \`{e}
espresso in secondi e $x(t)$ \`{e} espresso in metri, il prodotto delle dimensioni
di $5$ e $t$ deve restituire $m$. Quindi $5$ deve essere rappresentato in $\frac{m}{s}$.\\
Abbiamo quindi capito che il $5$ \`{e} una velocit\`{a}\\
Un punto fermo ha sempre velocit\`{a} 0.\\
Se prendiamo due posizioni $x_1, x_2$, possiamo determinare lo spostamento dell'oggetto
$\Delta x=x_2-x_1$. Il segno della velocit\`{a} indica se ci si sta spostando "avanti"
o "indietro" in base al passaggio del tempo.\\
Una importante osservazione da fare riguardo allo spostamento \`{e} 
che se si cambia l'origine della retta, lo spostamento di due punti $x_1, x_2$ rimane uguale.\\
Se usiamo un'origine $O$ in cui $x_1=3 m, x_2=5 m$, possiamo cambiare l'origine 
in $O'$ dove $x_1= 2m, x_4 = 4m$ e notiamo che $\Delta x= \Delta x' = x_2-x_1 = x_1'-x_2'$.\\
Questo ci aiuter\`{a} a definire la velocit\`{a} (spostamento nel tmepo).\\
\begin{tabular} {c | c}
    $t(s)$ & $x(m)$\\
    \hline
    0 & 1\\
    1 & 2\\
    4 & 1\\
    5 & 3,5
\end{tabular}
Possiamo dire che dall'istante 0 all'istante 1:\\
$\Delta x=1m, \Delta t=1s$\\
Possiamo anche dire che la velocit\`{a} media $\frac{\Delta x}{\Delta t}$\\
Nel nostro caso quindi la velocit\`{a} media $v_{media}=\frac{1m}{s}$.\\
La velocit\`{a} in cinematica ha dimensione $[V]=[LT^{-1}]$.\\
Possiamo calcolare la velocit\`{a} media per una coppia arbitraria di istanti, es:\\
$t=1$ e $t=4$\\
$v_{media}=\frac{(1-2)m}{(4-1)s}=-\frac{1m}{3s}\approx 0,33\frac{m}{s}$\\
Possiamo ricavare che $\Delta x = v_{media}\Delta t$.\\
\section*{7 Marzo 2018}
\subsection*{Moto rettilineo uniforme e moto rettilineo variabile}
La velocit\`{a} pu\`{o} essere caratterizzata in velocit\`{a} media ed istantanea.\\
$v_{media}=\frac{\Delta x}{\Delta t}$.\\
$v$ (istantanea) $=\lim_{\Delta t\to 0}\frac{\Delta x}{\Delta t}=\frac{dx}{dt}$ (derivata in rispetto a $t$).\\
Consideriamo una $v$ costante, allora:\\
$v_{media}=\frac{\Delta x}{\Delta t} = v$, possiamo allora dire che $\Delta x = v\Delta t$.\\
Detto ci\`{o}, possiamo fare una legge oraria anche della velocit\`{a}, oltre che della posizione.\\
Data questa legge oraria, possiamo determinare lo spostamento $\Delta x$ tramite il prodotto $v\cdot \Delta t$, che corrisponde
all'area sottesa al grafico nell'intervallo di tempo voluto.

\end{document}
